\chapter{Preliminaries}

\section{Linicrypt}
\subsection{Definition of a Linicrypt program}

The Linicrypt model for cryptographic constructions was introduced by Cramer \& Rosulek in \cite{C:CarRos16}.
Summarizing the formalization from that paper,
a pure Linicrypt program $\P$ is a straight line program
whose intermediate variables are elements in a field $\F$.
% TODO maybe operation isn't right
The only allowed operations to create an intermediate variable are:
\begin{itemize}
  \item Retrieve an input, which is in $\F$
  \item Perform a linear combination of existing internal variables
  \item Call a random oracle $\H: \{0,1\}^* \times \F^* \to \F^*$
  \item Sample from $\F$ uniformly
\end{itemize}

The program $\P$ can output one or more of its variables.

Below is an example of a Linicrypt program $\P^\H$,
% TODO Linicrypt is actually a tuple of CMD's ... should I specify that? Maybe in the 
% block cipher part.
written in conventional pseudocode on the left and in explicit Linicrypt on the right. 

\begin{pchstack}[center, space=0.4cm]
  \pcbl[valign=c]{$\P^\H(x, y)$}{
    r \sample \F \\
    \pcreturn \H(x+r) + y
  }
  \pseudocode[valign=c]{\rightsquigarrow}
  \pcbl[valign=c]{$\P^\H(x, y)$}{
    v_1 \deq x \\
    v_2 \deq y \\
    v_3 \sample \F \\
    v_4 \deq v_1 + v_3 \\
    v_5 \deq \H(v_4) \\
    v_6 \deq v_5 + v_2 \\
    \pcreturn (v_6)
  }
\end{pchstack}

\subsection{Type of Adversaries}
The Linicrypt model only imposes computational restrictions on the constructions,
not on the adversaries.
Usually one considers arbitrary adversaries $\A$ that are computationally unbounded,
but have bounded access to the random oracle $\H$.
Therefore the behaviour of an adversary is typically described in terms of the number of queries it makes.

\subsection{Algebraic Representation}

One of the advantages of restricting the computational model is that one can characterize
Linicrypt programs with an algebraic representation.
Let $\P$ be a linicrypt program with intermediate variables $v_1, \dots, v_n$.
These are sorted in the order in which they are created in the program.

A \textbf{base variable} is an intermediate variable which was created by retrieving an input,
calling the random oracle $\H$ or sampling from $\F$.
These are special, because they can be independent of each other.
Let $\base$ be the number of base variables.
We denote by $\vbase \in \Base$ the column vector
consisting of the values that the base variables take in a specific excecution of $\P$.
Here, we order the base variables the same way as the intermediate variables.
A \textbf{derived variable} is one which is created by performing a linear combination of existing itermediate variables.
Note, that derived variables are therefore linear combinations of base variables.
As base variables are mostly independent of each other,
it makes sense to \emph {model them as linearly independent vectors in $\Base$}.
The derived variables are then modeled by linear combinations of these vectors.

Let $v_i$ be an intermediate variable.
% TODO here we make the distinction betwenn variable, it's value on an execution, is this cleaner? 
We define the \textbf{associated vector} $\v v_i$ to be the unique row vector such that
for every excecution of $\P$ base variables taking the values $\vbase$,
the variable $v_i$ has the value $\vv_i \vbase$.
For example, if $v_i$ is the j'th base variable, then 
$
  \v v_i = \begin{bmatrix}
  0 & \cdots & 1 & \cdots & 0
  \end{bmatrix},
$
where the $1$ is in the j'th position.
We follow the convention to write matrices and vectors using a bold font.

The outputs of $\P$ can be described by a matrix with entries in $\F$.
Let $o_1, \dots, o_l \in \{v_1, \dots, v_n \}$ be the output variables of $\P$.
Then the \textbf{output matrix} $\O$ of $\P$ is defined by
\[
  \O =
  \begin{bmatrix}
  \v o_1\\
  \vdots \\
  \v o_l
  \end{bmatrix}
  .
\]
By the definition of the associated vectors $\v o_i$ we have
$
\O \vbase = 
  \begin{bmatrix} o_1 & \cdots & o_k \end{bmatrix}^\top
$.
The output matrix describes the linear correlations in the output of $\P$.

In the same way we also define the \textbf{input matrix} of $\P$.
If $i_1, \dots, i_k \in \{v_1, \dots, v_n\}$ are the intermediate variables created by retrieving an input,
then we write
\[
  \Inp = \begin{bmatrix}
  \v i_1 \\
  \vdots \\
  \v i_k
  \end{bmatrix}.
\]
As $i_1, \dots, i_k$ are base variables,
the rows of $\Inp$ are canonical basis vectors.
If the Linicrypt program is written such that it first retrieves all its inputs,
then $\v i_m$ is simply the $m$'th canonical basis vector. 

The input and output matrix describe the linear correlations between input and output of the program with respect to it's base variables.
But the base variables are not completely independent of each other.
In fact, the relationship between the queries and answers to the random oracle $\H$ needs to be captured algebraically.
Let $a_i = H(t_i, (q_1, \dots, q_n))$ be an operation in $\P$.
The \textbf{associated oracle constraint} $c$ of this operation is the tuple
\[
  c = \left( t_i, \begin{bmatrix}
  \v q_1 \\
  \vdots \\
  \v q_n
  \end{bmatrix},
  \va_i \right)
	=
	(t_i, \Q_i, \va_i)
	.
\]
This should be interpreted as the requirement that
$
\va_i \vbase = \H( t_i, \Q_i \vbase)
$.
We denote the set of all (associated) oracle constraints of $\P$ by $\C$.

As we want the base variables to be linearly indepentent from each other,
we restrict ourselves to Linicrypt programs which don't make multiple calls to the random oracle with the same input.
In the language of the algebraic reprensentation:
We assume wlog that no two constraints in $\C$ share the same $(t, \Q)$.

Wrapping up these definitions,
we define the \textbf{algebraic representation} of $\P$ to be the tuple $(\Inp, \O, \C)$.
A natural question that arises at this point is:
Does the algebraic representation determine the behaviour of $\P$ completely?

The answer is yes, because the algebraic representation does not loose any relevant information about the operations executed in $\P$.
Informally, this is because the constraints in $\C$ have a particular form, which makes it clear in which order the oracles calls have to be excecuted.
Given the order, and using the input matrix, one can determine which variables used in the calls are retrieved from the input and which have to be sampled.
Finally the output matrix completely describes how to construct the output from the input,
the results of the queries and randomly sampled values. 

% TODO mention further result of equivalence when in normalized form
The authors of the original paper on Linicrypt \cite{C:CarRos16} establish this result for inputless programs

\subsection{Characterizing Collision Resistance in Linicrypt}

In a paper by I. McQuoid, T. Swope and M. Rosulek
\cite[Characterizing Collision and Second-Preimage Resistance in Linicrypt]{TCC:McQSwoRos19},
the authors introduced a characterization of collision resistance
and second-preimage resistance for a class of Linicrypt program based on the algebraic representation.

They identified two reasons why a \textit{deterministic} Linicrypt $\P$ program can fail to be second-preimage resistant:
\begin{enumerate}
  \item It is degenerate, meaning that it doesn't use all of its inputs independently
  \item It has a collision structure,
    which means that one can change some intermediate value and compute what the input needs to be to counteract this change
\end{enumerate}

Below are two example Linicrypt programs, $\P^\H_\textrm{deg}$ is degenerate and $\P^\H_\textrm{cs}$ has a collision structure.
Note, that you can choose $w' \neq w$ to be any value,
then find a $x'$ such that the output of $\P^\H_{\textrm{cs}}$ stays the same,
and finally find $y'$ according to $w' = x' + y'$.

\begin{pchstack}[center,space=2cm]
  \pcbl[valign=c]{$\P^\H_{\textrm{deg}}(x, y)$}{
    v \deq x + y \\
    \pcreturn H(v)
  }
  \pcbl[valign=c]{$\P^\H_{\textrm{cs}}(x, y)$}{
    w \deq x+y \\
    \pcreturn \H(w) + x
  }
\end{pchstack}

The precise definition of degenerate and collision structure will be discussed in section \ref{section_collision_structure}.
The authors show that for any deterministic Linicrypt program which is degenerate
or has a collision structure
second-preimage resistance (and hence also collision resistance) is completely broken.

The main result of \cite{C:CarRos16} is that they show that the converse of this is also true,
for Linicrypt programs which use distinct nonces in each call to the random oracle.
That is, if a Linicrypt program is not collision resistant,
then it either has a collision structure, or it is degenerate.

Furthermore, checking for degeneracy and existence of a collision structure can be done efficiently.
