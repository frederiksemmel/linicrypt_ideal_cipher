\chapter{Linicrypt with Ideal Ciphers}

\section{Revisiting Algebraic Representations}
We have defined the algebraic representation for a Linicrypt program.
It consists of the input matrix $\Inp \in \F^{k \times \base}$,
the output matrix $\O \in \F^{l \times \base}$
and the set of constraints $\C = \{ (t_i, \Q_i, \va_i) \mid i = 1, \dots, n \}$,
with $\Q_i \in \F^{m_i \times \base}$ and $\va_i \in \F^{1 \times \base}$.
These matrices are such,
that if one constraints $\vbase$ via $\I\vbase = \m{i_1 & \cdots & i_k}^\top$ for some arbitrary input,
the program can compute the rest of the base variables and output $\O \vbase = \m{o_1 & \cdots & o_l}^\top$.
To simplify the notation we will also write $\vec i k$ for the column vector $\m{i_1 & \cdots & i_k}^\top$.

Our goal is to find a better understanding of collision structures.
The algebraic representation suggest,
that input and output matrix play the same role, they constrain the base variables. 
Therefore properties of collision resistance should be expressible in terms of the ability to solve the constraints $\C$ while setting $\O\vbase$ to some value.  
This idea will be formalized in this section.
A question that arises is:
Which combination of matrices with the structure described above correspond in essence to a Linicrypt program?
To answer this question, we need to define the terminology more carefully.

\begin{defn}[Random oracle constraint]
A \textbf{random oracle constraint} of dimension $\base$ taking $m$ inputs is a tuple $(t, \v Q, \v a)$ for
$t \in \bin^*$, $\Q \in \F^{m \times \base}$ and $\va \in \F^{1 \times \base}$.
\end{defn}

We call $t$ the nonce and refer to $\Q$ and $\va$ as the query and answer to the random oracle.
Usually we just say ``a constraint" when the other variables are clear from the context.
The constraint $(t, \Q, \va)$ encodes the relationship via the random oracle between the base variables in a program.
This semantic meaning of constraints is encoded in the following definition.
\begin{defn}[Solution of constraints]
    Let $\C$ be a set of constraints of dimension $\base$.
    We say a vector $\vv \in \Base$ \textbf{solves} $\C$ if
    $\va \vv = \H(t, \Q \vv)$ for all $(t, \Q, \va) \in \C$.
    Such a $\vv$ is also called a \textbf{solution} of $\C$.
    % TODO maybe include here sol(C) = all v that solve C
\end{defn}

Clearly, by the definition of the algebraic representation,
any instantiation $\vbase$ of the base variables in a Linicrypt program is a solution to $\C$.

Because $\H$ is a well-defined function,
and not just any relation,
these requirements extend to the constraints.

\begin{defn}[Well-defined]
A set of (random oracle) constraints $\C$ is \textbf{well-defined} if for any pair of constraints 
$c_i, c_j \in \C$ we have $(t_i,\Q_i) = (t_j, \Q_j) \implies \va_i = \va_j$.
\end{defn}

When we use a set of constraints, we will implicitly also require that it is well-defined.
Highlighting the function like properties of a constraint $(t, \Q, \va)$,
we will also use the shorthand $(t,\Q) \mapsto \va$, or $\Q \mapsto \va$ if $t$ in case t is an empty string.
This notation was introduced in \cite{CSF:HolRosRoy22}.
If a set of constraints is well-defined, it might still not correspond to a valid Linicrypt program.
Consider the set
\[
\C = \left\{ \m{1&0} \mapsto \m{0&1}, \m{0&1} \mapsto \m{1&0}\right\}.
\]
Although it is well-defined,
it corresponds to the two calls to the random oracle that are not compatible with the randomness assumptions of the oracle.
Specifically, such constraints would correspond to a program which does
$y = H(x)$ and $x = H(y)$, where $x,y$ are intermediate variables.
This is not possible in the Linicrypt model.

In order to characterize which constraints can be solved by a Linicrypt program,
we also need to consider the concept of basis change which was introduced in \cite{C:CarRos16}.

\begin{defn}[Basis change]
    Let $\v B$ be any matrix in $\F^{\base \times \base}$
    and $\P = (\Inp, \O, \C)$ be a Linicrypt program.
    We define 
    \[
        \C \B \coloneqq \big\{ (t, \Q \B, \va \B) \mid (t, \Q, \va) \in \C \big\}.
    \]
    If $\B$ is invertible,
    we say $(\Inp \B, \O \B, \C \B)$ is the algebraic representation of $\P$
    after applying the \textbf{basis change} $\B$.
\end{defn}

In the case that $\B$ is just a permutation of the input variables
the transformed representation clearly describes the same program in an informal sense,
we have just renamed the input variables.
But a more complex basis changes won't change the program either, as is shown below.

Now we can characterize which sets of constraints and input matrices correspond to a Linicrypt program.

\begin{defn}[Deterministically Solvable]
\label{def_det_solvable}
Let $\C$ be a finite set of valid constraints of dimension $\base$,
and let $\I \in \F^{k \times \base}$ for some $k \in \NN$.
$\C$ is \textbf{deterministically solvable} fixing $\I$
if there exists an ordering $(c_1, \dots, c_n)$ of $\C$
such that for all $i=1, \dots n$:
\begin{enumerate}
\item
    \label{solvable1}
    $\rows(\Q_i) \subset \spn \big(\rows(\I) \cup \{\va_1, \dots, \va_{i-1}\}\big)$
\item
    \label{solvable2}
    $\va_i \notin \spn \big(\rows(\I) \cup \{\va_1, \dots, \va_{i-1}\} \cup \rows(\Q_i) \big)$
\end{enumerate}
Additionally we require that $\rows(\I) \cup \{\va_1, \dots, \va_n\}$ form a basis of $\F^{1\times\base}$.
We call $(c_1, \dots, c_n)$ a solution ordering of $\C$ fixing $\I$ (or fixing $\rows(\I)$).
\end{defn}

This is a similar to the definition of collision structure in \cite{TCC:McQSwoRos19}.
We will use it to generalize the collision structure and degeneracy attacks from that work.

Intuitively, $\C$ being deterministically solvable fixing $\I$ means,
that we can constrain $\vbase$ by $\Inp \vbase = (i_1, \dots, i_k)$ for arbitrary inputs $i_1, \dots, i_k \in \F$,
and then compute each oracle query (condition \ref{solvable1}) one by one,
without creating a contradiction (condition \ref{solvable2}).


% TODO in general it is unclear if P is deterministic or not. I think everything works for
% non deterministic programs too, just more cumbersome and useless for my purpose.
If we construct the algebraic representation $(\I, \O, \C)$ of a deterministic Linicrypt program $\P$,
then $\C$ is a deterministically solvable set of constraints fixing $\I$.
Indeed, the solution ordering of $\C$ fixing $\I$
can be exactly the order of the corresponding queries in the execution of $\P$.

\begin{lemma}
    Let $(\I, \O, \C)$ be the algebraic representation of a deterministic Linicrypt program $\P$ taking $k$ inputs.
    Then $\C$ is deterministically solvable fixing $\I$.
\end{lemma}

\begin{proof}
    The proof works by induction.
    Consider a constraint $(t, \Q, \va)$ in $\C$.
    By definition, for some $i$,
    $\va$ is the $i$'th canonical basis row vector,
    and $\Q$ only has only zeros in the columns to the right of column $i$.
    We can sort $\C = \{\dts c n\}$, such that the $\va$'s are sorted.
    Indeed, this is the same order as the associated oracle calls in the execution of $\P$.
    
    Clearly condition \ref{solvable2} from definition $\ref{def_det_solvable}$ is then fulfilled.
    Because $\P$ is deterministic, 
    the queries to the oracle are linear combinations of input variables and results of previous queries.
    Therefore condition \ref{solvable1} must be fulfilled.
    Finally, as $\P$ contains no sampling operation, we know $\base = k + n$.
    The associated vectors $\rows(I) = \dts \vi k$ to the input variables are linearly independent,
    hence it follows that $\{ \dts \vi k, \dts \va n \}$ is a basis of $\F^{1\times\base}$. 
\end{proof}

The following Lemma shows that the reversed direction also works,
meaning that we can recover a Linicrypt program from solvable constraints by applying a basis change.
The equivalence up to basis change is not a problem, 
because any characterization we make about the algebraic representation will always be invariant under basis change.

\begin{lemma}
\label{det_solvable_lemma}
    Let $\C$ be a set of deterministically solvable constraints fixing $\Inp \in \F^{k \times \base}$ for some $k \in \NN$.
    Let $\O \in \F^{\out \times \base}$ be an arbitrary output matrix for some $\out \in \NN$.
    Then there is a basis change $\B \in \F^{\base \times \base}$
    and a Linicrypt program $\P$,
    such that $(\Inp \B, \O \B, \C \B)$ is it's algebraic representation.
\end{lemma}

\begin{proof}
Let $(c_1, \dots, c_n)$ be the solution ordering of $\C$ fixing $\I = \mtdts \vi k$.
We choose the new basis for $\Base$ as $(\vi_1, \dots, \vi_k, \va_1, \dots, \va_n)$, hence the basis change matrix is defined by
\[
    \B^{-1} = 
    \begin{bmatrix}
\vi_1 &
\cdots &
\vi_k &
\va_1 &
\cdots &
\va_n
    \end{bmatrix}^\top.
\]

In the following we denote by $\v e_i$ the $i$'th canonical row vector.
By definition of $\B$ we have for $i=1, \dots, n$
% TODO this is not compact enough, for its importance
\begin{equation*}
\I = \begin{bmatrix}
    \v e_1 \\
    \vdots \\
    \v e_k
\end{bmatrix} \B^{-1}
\qquad \textrm{and} \qquad
\va_i = \v e_{k+i} \B^{-1},
\end{equation*}
which is equivalent to
\begin{equation*}
\I \B = \begin{bmatrix}
    \v e_1 \\
    \vdots \\
    \v e_k
\end{bmatrix}
\qquad \textrm{and} \qquad
\va_i \B = \v e_{k+i}.
\end{equation*}

Additionally, as $\C$ is solvable via the ordering $(c_1, \dots, c_n)$,
we know that 
\[
\rows(\Q_i \B) \subset \spn(\vi_1 \B, \dots, \vi_k \B, \va_1 \B, \dots, \va_{i-1} \B) = \spn\big( \ve_1, \dots, \ve_{k+i-1} \big)
\]

In other words, the query and answer matrices for the constraints in $\C \B$ have the form
\begin{align*}
\Q_i \B &= \begin{bmatrix} \parbox{12pt}{$\lambda^i_1$} & \cdots & \parbox{32pt}{$\lambda^i_{k+i-1}$} & 0 & 0 & \cdots & 0 \end{bmatrix} \qquad \textrm{for} \quad \lambda^i_j \in \F \\
\va_i \B&= \begin{bmatrix} \parbox{12pt}{0}             & \cdots & \parbox{32pt}{0}                 & 1 & 0 & \cdots & 0 \end{bmatrix}
\end{align*}

Both the input matrix $\I \B$ and $\C \B$ have the correct form to be 
the algebraic representation of a program $\P$ taking $k$ inputs,
and outputting according to $\O \B$.
\end{proof}

In order to abstract the questions about deterministic Linicrypt programs to the level of the algebraic representation,
we need the following Lemma.

\begin{lemma}
\label{det_solvable_equiv}
    Let $\P = (\I, \O, \C)$ be a deterministic Linicrypt program.
    The following two statements are equivalent:
    \begin{enumerate}
    \item
        Running $\P$ on input $(i_1, \dots, i_k)$ gives output $(o_1, \dots, o_l)$.
    \item 
        There exists a $\vv$ solving $\C$ such that
        $\I\vv = \m{i_1 \\ \vdots \\ i_k}$ and
        $\O\vv = \m{o_1 \\ \vdots \\ o_k}$.
    \end{enumerate}
\end{lemma}

\begin{proof}
    Direction 1. $\implies$ 2. is clear, by definition of the algebraic representation.
    In fact, such a $\vv$ is unique, as $\P$ is deterministic.
    
    For the other direction, assume we have such a $\vv \in \Base$.
    If we run $\P$ on $(i_1, \dots, i_k)$ we compute the base variables $\vv' \in \Base$.
    By definition of the algebraic representation we have $\I \vv' = \vec i k = \I \vv$.
    Also, $\vv'$ solves $\C$, by definition.
    We will prove that $\vv = \vv'$.

    We know $\C$ is deterministically solvable fixing $\I$, hence
    $\rows(\Q_1) \in \spn\{ \rows(\I)\}$. 
    In other words, there is a $\v \lambda$ such that $\Q = \vl \I$.
    Therefore we have $\Q_1 \vv' = \vl \I \vv' = \vl \I \vv = \Q_1 \vv$.
    Using that both $\vv'$ and $\vv$ solve $\C$ gives
    $\va_1 \vv' = \H(\Q_1 \vv') = \H(\Q_1 \vv) = \va_1 \vv$.
    Continuing the induction, we get that $\va_i \vv' = \va_i \vv$ for $i = 1, \dots, n$.
    Because $\rows(\I) \cup \{\va_1, \dots, \va_n\}$ is a basis, 
    the matrix 
    \[
    \B = \m{\I \\ \vdts \va n}
    \]
    is invertible.
    Using the assumptions and the results from the induction we have shown that $\B \vv' = \B \vv$ and
    therefore $\vv' = \vv$.
    Because this means $\O \vv' = \O \vv = \vec o l$,
    we have shown that running $\P$ on input $(i_1, \dots, i_k)$ gives output $(o_1, \dots, o_l)$.
\end{proof}


\section{Revisiting Collision Structures}

Using this language we can argue about the invertibility and second preimage resistance of a Linicrypt program.
The goal of this section is to describe collision structures in a more abstract form, 
which is easier to adapt to the case of using ideal ciphers instead of random oracles.
We will describe collision structures as a program being partially invertible with extra degrees of freedom.
As a stepping stone towards this goal, we start by describing a sufficient condition for a Linicrypt program to be invertible.

Let $\P = (\I, \O, \C)$ be a deterministic Linicrypt program taking $k$ inputs and returning $l$ outputs.
First we note that one can see a deterministic Linicrypt program as a function from its input space to its output space.
We associate to $\P$ the function $f_\P: \F^k \to \F^l$ which maps $(\dts i k)$ onto the output of $\P(\dts i k)$.

\begin{lemma}
\label{inversion}
    $\P = (\I, \O, \C)$ be a deterministic Linicrypt program.
    Assume $\C$ is deterministically solvable fixing $\O$.
    Then $f_\P$ is bijective and $f_\P^{-1}$ is the associated function of a deterministic Linicrypt program which we call $\P^{-1}$.
\end{lemma}

\begin{proof}
By Lemma \ref{det_solvable_lemma} we get a Linicrypt program $\P^{-1}$ with algebraic representation $(\O \B, \I \B, \C \B)$.
Note, that $\O$ and $\I$ have the same number of rows, because both $\rows(\I) \cup \{\dts \va n\}$ and
$\rows(\O) \cup \{\dts \va n\}$ are bases of $\F^{1\times\base}$.

By associativity of the matrix product, we have the following equivalence:
\begin{gather*}
    \vv \textrm{ solves } \C \B \textrm{ with } (\I \B) \vv = \vec i k \textrm{ and } (\O \B) \vv = \vec o l \\
    \Updownarrow \\
    \B \vv \textrm{ solves } \C \textrm{ with } \I (\B \vv) = \vec i k \textrm{ and } \O (\B \vv) = \vec o l 
\end{gather*}

Combining this equivalence with the one from Lemma \ref{det_solvable_equiv} we get:
Running $\P$ on input $\dts i k$ giving output $\dts o l$ is equivalent to running 
$\P^{-1}$ on input $\dts o l$ with output $\dts i k$.
This means that $f_{\P^{-1}} = f_\P^{-1}$, which is what we needed to prove.
\end{proof}

For example, the Lemma applies to the following program $\P$. 
The proof constructs the algebraic representation of the inverse program $\P^{-1}$.
\begin{pchstack}[center,space=2cm]
    \pcbl[valign=c]{$\P(x,y)$}{
        \pcreturn (y + \H(x), x)
    }
    \pcbl[valign=c]{$\P^{-1}(w,z)$}{
        \pcreturn (z, w - \H(z))
    }
\end{pchstack}

To simplify the following statement about second preimages we introduce a shorthand for all the row vectors contained in a set of constraints $\C$.
We define \[
\rows(\C) = \bigcup_{(\Q,\va) \, \in \, \C} \big\{ \rows(\Q) \cup \{ \va \} \big\}.
\]

\begin{lemma}
    Let $\P = (\I, \O, \C)$ be a deterministic Linicrypt program.
    Assume we can write $\C = \Cfix \cup \Ccs$ and
    $\Ccs$ is deterministically solvable fixing some $\I_{cs}$ such that
    \begin{equation}
    \label{cs_condition}
        \spn(\rows(\I_\textrm{cs})) \supset \spn\big(\rows(\Cfix) \cup \rows(\O)\big).
    \end{equation}
    Then an attacker can find second preimages with $\leq n$ queries.
\end{lemma}

This Lemma is similar to Lemma \ref{inversion} in the sense that both describe a condition for being able to find a $\vv$ solving $\C$ while fixing $\O\vv = \vec o l$.
The difference is that here we consider the case where we already have a solution $\vv'$ with $\O \vv' = \vec o l$,
and require a different one. 

Note, that it is crucial that the space on the left is $\supset$ and not only $\supseteq$,
as this gives the extra degree of freedom to find a different preimage.
It plays the same role as $\vq_{i^*}$ plays in the definition of collision structure from \cite{TCC:McQSwoRos19}.
This characterization includes the case of degeneracy from \cite{TCC:McQSwoRos19},
namely it corresponds to choosing $\Ccs = \{\}$ and $\I_\textrm{cs} = \Id_\base$,
while degeneracy means precisely $\F^{1\times\base} \supset \spn(\rows(\C) \cup \rows(\O))$.

\begin{proof}
Assume we are given a $\vv'$ solving $\C$ such that $\I \vv' = \vec i k$ and $\O \vv' = \vec o l$.
The goal is to find a different $\vv$ solving $\C$ with $\O \vv = \vec o l$.
By \eqref{cs_condition} we can constrain $\vv$ to fulfill $\O \vv = \O \vv'$ and ($\Q \vv = \Q \vv'$ and $\va\vv = \va\vv'$) for every $(t, \Q, \va) \in \Cfix\,$.
Even doing so, we still have at least one degree of freedom before we have uniquely determined what
$\I_{cs} \vv$ is.
We can therefore choose these further constraints to be arbitrary such that $\vv \neq \vv'$.
Because $\Ccs$ is deterministically solvable fixing $\I_{cs}$,
we can uniquely determine all the components of $\vv$ such that they solve $\Ccs$.
But we chose $\I_{cs} \vv$ in such a way, that $\vv$ also solves $\Cfix\,$,
hence $\vv$ solves $\C = \Ccs \cup \Cfix$.

Finally, because $\P$ is deterministic, $\vv \neq \vv'$ implies $\I \vv \neq \I \vv'$,
so $\I\vv$ is a second preimage to $\I\vv'$ for the Linicrypt program $\P$.
\end{proof}

The main benefit of this perspective on the collision structure attack is,
that it directly describes the collision structure attack as a Linicrypt program.
If we apply Lemma \ref{det_solvable_lemma} to $\Ccs$ while fixing $\I_{cs}$,
the resulting ``attack program" is one taking as input the desired output,
some values determined by the execution on the given preimage,
and some extra values which can be set freely.

The following is the running example from \cite{TCC:McQSwoRos19}.
\pcb[valign=c]{$\P(x,y,z)$}{
    w = \H(x) + \H(z) + y \\
    \pcreturn (\H(w) + x, \H(z))
}
% TODO fix alignment
\begin{align*}
\I    &= \begin{bmatrix}
            1 & 0 & 0 & 0 & 0 & 0 \\
            0 & 1 & 0 & 0 & 0 & 0 \\
            0 & 0 & 1 & 0 & 0 & 0 
            \end{bmatrix} \\
\O    &= \begin{bmatrix}
            1 & 0 & 0 & 0 & 0 & 1 \\
            0 & 0 & 0 & 0 & 1 & 0 
            \end{bmatrix} \\
\vq_1 &= \begin{bmatrix}1 & 0 & 0 & 0 & 0 & 0\end{bmatrix} \\
\va_1 &= \begin{bmatrix}0 & 0 & 0 & 1 & 0 & 0\end{bmatrix} \\
\vq_2 &= \begin{bmatrix}0 & 0 & 1 & 0 & 0 & 0\end{bmatrix} \\
\va_2 &= \begin{bmatrix}0 & 0 & 0 & 0 & 1 & 0\end{bmatrix} \\
\vq_3 &= \begin{bmatrix}0 & 1 & 0 & 1 & 1 & 0\end{bmatrix} \\
\va_3 &= \begin{bmatrix}0 & 0 & 0 & 0 & 0 & 1\end{bmatrix} \\
\end{align*}

We set $\Ccs = \{ c_1, c_3 \}$ and $\Cfix = \{c_2\}$.
Then $(c_3, c_1)$ is a solution ordering for $\Ccs$ fixing 
\[
\I_{cs} = \begin{bmatrix}\O & \vq_2 & \vq_3 \end{bmatrix}^\top.
\]
Note, that for this matrix we have the required condition \eqref{cs_condition}, because
\[
\spn(\rows(\I_{cs})) \supseteq
\rows(\O) \cup \rows(\Cfix) \cup \{ \vq_3 \},
\]
and $\vq_3$ is linearly independent from the rest on the left hand side.
Then Lemma \ref{det_solvable_lemma} gives us the transformation $\B$
such that $(\I_{cs} \B, \I \B, \Ccs \B)$
is the algebraic representation of some $\P_{cs}$.
Here we have
% TODO fix alignment
\[
\B = 
\begin{bmatrix}
1 & 0 & 0 & 0 &-1 & 0 \\
0 &-1 & 0 & 1 & 0 &-1 \\
0 & 0 & 1 & 0 & 0 & 0 \\
0 & 0 & 0 & 0 & 0 & 1 \\
0 & 1 & 0 & 0 & 0 & 0 \\
0 & 0 & 0 & 0 & 1 & 0
\end{bmatrix},
\]
which leads to the representation:
\begin{align*}
    % TODO fix this constraints mess
    \C_{cs} \B &= \begin{Bmatrix}
        \m{0 & 0 & 0 & 1 &\phantom{-}0 & 0} \mapsto \m{0 & 0 & 0 & 0 & 1 & 0} \} \\
        \m{1 & 0 & 0 & 0 &-1 & 0} \mapsto \m{0 & 0 & 0 & 0 & 0 & 1} \}
    \end{Bmatrix} \\
    \I_\textrm{cs} \B &= \m{\Id_4 & 0} \\
    \I \B &= \begin{bmatrix}
1 & 0 & 0 & 0 &-1 & 0 \\
0 &-1 & 0 & 1 & 0 &-1 \\
0 & 0 & 1 & 0 & 0 & 0
    \end{bmatrix}    
\end{align*}

In the form of a Linicrypt program this is:
\pcb[valign=c]{$\P_{cs}(o_1,o_2,z, q_3)$}{
    a_3 = \H(q_3) \\
    v = \H(o_1 - a_3) \\
    \pcreturn (o_1 - a_3, q_3 -o_2 - v, z)
}

This program computes second preimages to the input $(x,y,z)$
if we set $(o_1, o_2) = \P(x,y,z)$ and choose $q_3$ arbitrarily.


\section{Adapting the Linicrypt model to use block ciphers}

In this chapter we modify the Linicrypt model to make use of the ideal cipher model instead of the random oracle model.
This means that a Linicrypt program gets access to a block cipher $\mathcal{E} = (E, D)$ where $E$ and $D$ are functions $\F \times \F \to \F$
instead of the hash function $\H : \{0,1\}^* \times \F^* \to \F$.
By the definition of a block cipher,
$\E_k := \E(k, \cdot)$ is a permutation of $\F$ for all $k \in \F$ and
$\D_k := \D(k, \cdot)$ is its inverse.
In the ideal cipher model, we assume that the block cipher has no weakness.
This is modeled by choosing each permutation $\E_k$ uniformly at random at the beginning of every security game.

The command $y = \E(k, x)$ in a Linicrypt program has to be treated differently from the command $y = \H(k, x)$ when considering collision resistance,
because an attacker has access to the deterministic Linicrypt program and both directions of the block cipher $\mathcal{E} = (\E, \D)$.
Consider these two programs, $\PH$ in standard Linicrypt and $\PE$ in ideal cipher Linicrypt.

\begin{pchstack}[center,space=2cm]
  \pcbl[valign=c]{$\PH(k, x)$}{
    \pcreturn \H(k,x)
  }
  \pcbl[valign=c]{$\PE(k, x)$}{
    \pcreturn \E(k,x)
  }
\end{pchstack}
While $\PH$ is collision resistant, it is trivial to find second preimages for $\PE$
For any $k' \in \F$ the pair $(k', \D(k', \E(k, x)))$ is a second preimage to $(k,x)$.

This invertibility property of block ciphers has to be taken into account
in both the algebraic representation and the characterization of collision resistance.

\subsection{Algebraic representation for ideal cipher Linicrypt}

Let $\P$ be a ideal cipher Linicrypt program. For each query to $\E$ of the form $y =
\E(k,x)$ we define the \textbf{associated ideal cipher constraint} $(\E, \v k, \v x, \v y)$.
Each query to $\D$ of the form $x = \D(k, y)$, is
associated with the constraint $(\D, \v k, \v y, \v x)$.

As with standard Linicrypt,
we want to exclude programs that make unnecessary queries to the block cipher.
This way the base variables are linearly independent,
except for the dependencies the adversary might introduce by carefully choosing the input.
Hence we assume wlog that no two constraints have the same $(\E, \v k, \v x)$ or $(\D, \v k, \v y)$.

With ideal ciphers there is a second way to make an unnecessary query.
That is by first computing $y = \E(k,x)$ and then $x' = \D(k, y)$.
As $\D_k$ is the inverse of $\E_k$ we have $x = x'$ although $\v x$ and $\v x'$ are linearly independent.


Therefore for all $\v k, \v x, \v x', \v y, \v y' \in \Base$ we can assume there are no two constraints
$(\E, \v k,\v x, \v y)$ and $(\D, \v k, \v y, \v x')$ in $\C$ for $\v x \neq \v x'$.
Neither can there be $(\D, \v k,\v y, \v x)$ and $(\E, \v k, \v x, \v y')$ in $\C$ for $\v y \neq \v y'$.

TODO: Maybe it is simpler with equivalence relation called always colliding queries
\begin{align*}
    (\E, \v k,\v x, \v y) &\sim (\E, \v k, \v x, \v y') \\ 
    (\E, \v k,\v x, \v y) &\sim (\D, \v k, \v y, \v x') \\
    (\D, \v k,\v y, \v x) &\sim (\D, \v k, \v y, \v x') \\
    (\D, \v k,\v y, \v x) &\sim (\E, \v k, \v x, \v y')
\end{align*}
And saying that no two constraints in $\C$ are in the same equivalence class.
This might be cleaner, if the equivalence relation used later to analyze repeated nonces case is defined similarly.

\section{Collision Structure}

To explain the concept of a collision structure, we will make use of an example.
Consider the following Linicrypt program:
\pcb[valign=c]{$\PE[col,1](a,b,c)$}{
    w = E(c, b+c) + a \\
    \pcreturn c + \E(w,b)
}
A second preimage to $(a,b,c)$ can be found by the following procedure:
Choose some $w' \neq w$.
It will turn out, that even choosing $w'$ at random,
one can calculate what the other base variables need to be such that the output stays the same.
As we want $c + \E(w,b) = c' + \E(w', b')$ we can again choose any $b'$ and compute $c'$. 
Finally, we can compute $a'$ from the equation $w' = \E(c', c' + b') + a'$

One can more easily see that such a procedure is possible by looking at the algebraic representation of $\PE[col,1]$.
In order to highlight which are the base variables, we rewrite the program a bit more explicitly.
\pcb[valign=c]{$\PE[col,1](a,b,c)$}{
    k_1 = c \\
    x_1 = b+c \\
    y_1 = \E(k_1, x_1) \\
    k_2 = y_1 + a \\
    x_2 = b \\
    y_2 = \E(k_2, x_2) \\
    \pcreturn c + y_2
}
The base variables are $(a, b, c, y_1, y_2)$ and the algebraic representation is
\[
    \O = \begin{bmatrix} 0, 0, 1, 0, 1 \end{bmatrix} 
        \qquad \C = \{(\E, \v k_1, \v x_1, \v y_1), (\E, \v k_2, \v x_2, \v y_2)\}
\]
where
\begin{align*}
\v k_1 &= \begin{bmatrix} 0, 0, 1, 0, 0 \end{bmatrix} \\
\v x_1 &= \begin{bmatrix} 0, 1, 1, 0, 0 \end{bmatrix} \\
\v y_1 &= \begin{bmatrix} 0, 0, 0, 1, 0 \end{bmatrix} \\
\v k_2 &= \begin{bmatrix} 1, 0, 0, 1, 0 \end{bmatrix} \\
\v x_2 &= \begin{bmatrix} 0, 1, 0, 0, 0 \end{bmatrix} \\
\v y_2 &= \begin{bmatrix} 0, 0, 0, 0, 1 \end{bmatrix}.
\end{align*}

We can formulate the previous attack from in terms of the algebraic representation.
The task is to find a $\v v_\base' = [a', b', c', y_1', y_2'] \neq \v v_\base$ such that:
\begin{align}
\label{constraint_output}
\PE[col,1](a', b', c') = \O \vbase' &= \O \vbase = \PE[col,1](a,b,c) \\
\label{constraint_cipher_1}
y_1' = \vy_i \vbase' &=  \E(\vk_1 \vbase'\,, \; \vx_1 \vbase')\\
\label{constraint_cipher_2}
y_2' = \vy_i \vbase' &=  \E(\vk_2 \vbase'\,, \; \vx_2 \vbase')
\end{align}

We can fulfill these requirements step by step.
First, we constrain $\vbase'$ by requiring 
\[
    \O \vbase' = \O \vbase.
\]
This reduces the dimension of the space of possible solutions for $\vbase'$ to 4,
as $\O \vbase$ is in the range of $\O$ by definition.
Now note, that neither $\vk_2$ nor $\vx_2$ are in the span of $\rows(\O)$.
Therefore we can choose $k_2'$ and $x_2'$ at random such that $(k_2', x_2') \neq (k_2, x_2)$,
and constrain $\vbase'$ by requiring
\[
    \vk_2 \vbase' = k_2' \quad \textrm{and} \quad \vx_2 \vbase' = x_2'.
\]
Now we can compute $y_2' = \E(k_2', x_2')$ and add constraint \eqref{constraint_cipher_2}.
This constraint is compatible with the previous constraints because $\vy_2$ is not in the span of
$\rows(\O) \cup \{\vk_2, \vx_2\}$.
The dimension of the subspace of solutions is now 1, as we have added 4 one-dimensional constraints.

Now one only needs to fulfill constraint \eqref{constraint_cipher_1}.
As $\vk_1$ and $\vx_1$ are in the span of
$\rows(\O) \cup \{\vk_2, \vx_2, \vy_2\}$ the intermediate variables $k_1'$ and $x_1'$ are uniquely determined.
E.g. $\vk_1 = \O - \vy_2$ and therefore $k_1' = \PE[col,1](a,b,c) - y_2'$.

Finally, note that 
$\vy_1 \notin \spn\big( \rows(\O) \cup \{ \vk_2, \vx_2, \vy_2 \} \cup \{ \vk_1, \vx_1 \} \big)$.
Therefore, adding the constraint
\[
    \vy_1 \vbase' = y_1' = \E(k_1', x_1')
\]
reduces the solution space of possible $\vbase'$ to a single point in $\Base$.
We know that $\vbase' \neq \vbase$ because $(k_2, x_2') \neq (k_2, x_2)$.
The only way to produce different intermediate variables in a deterministic program is to choose different input,
hence $(a',b',c') \neq (a,b,c)$.

This example would have worked exactly the same if we replaced $\E$ with $\H$.
What follows is an example where the invertibility property of $\E_k$ plays a role.

\begin{pchstack}[center, space=2cm]
    \pcbl[valign=c]{$\PE[col,2](a,b,c)$}{
        k_1 = c \\
        x_1 = b \\
        y_1 = \E(k_1, x_1) \\
        k_2 = a \\
        x_2 = y_1 \\
        y_2 = \E(k_2, x_2) \\
        \pcreturn y_1 + y_2
    }
    \pseudocode[valign=c]{
        \O    = \begin{bmatrix}0,0,0,1,1\end{bmatrix} \\
        \vk_1 = \begin{bmatrix}0,0,1,0,0\end{bmatrix} \\
        \vx_1 = \begin{bmatrix}0,1,0,0,0\end{bmatrix} \\
        \vy_1 = \begin{bmatrix}0,0,0,1,0\end{bmatrix} \\
        \vk_2 = \begin{bmatrix}1,0,0,0,0\end{bmatrix} \\
        \vx_2 = \begin{bmatrix}0,0,0,1,0\end{bmatrix} \\
        \vy_2 = \begin{bmatrix}0,0,0,0,1\end{bmatrix} 
    }
\end{pchstack}
For this program there is a similar procedure to find second preimages.
As before, the first constraint is $\O $
In this case we can fix $\vk_2 \vbase' = k_2' = k_2$, $\vx_2 \vbase' = x_2' = x_2$ and $\vy_2 \vbase' = y_2' = y_2$.
Therefore condition \eqref{constraint_cipher_2} is fulfilled trivially.
After adding these 4 constraints the solutions space is still 1-dimensional.
Note, that $\vy_1 = \vx_2$ and it is therefore already fixed at this point,
hence to fulfill \eqref{constraint_cipher_1} we have to make use of the invertibility property of $\E_k$.
Because
\begin{align*}
    \vy_1 \vbase' &= \E(\vk_1 \vbase', \vx_1 \vbase') \\
    \iff \vx_1 \vbase' &= \D(\vk_1 \vbase', \vy_1 \vbase'),
\end{align*}
we can choose a random $k_1' \neq k_1$ and compute $x_1' = \D(k_1', y_1')$ in order to fulfill (3.1).
With this fifth constraint we have found a single $\vbase' \neq \vbase$.

We want to briefly summarize the conditions for this kind of second preimage attack.
Let $\P'$ denote the execution of 
\begin{itemize}
\item
    Some ideal cipher constraints are fulfilled by setting the intermediate variables in $\P(a',b',c')$ to the same value as in $\P(a,b,c)$ 
\item
    There is some constraint for which either the input or the output is undetermined by the previously fixed intermediate variables
\item
    For this constraint and all following constraint one can iteratively call the ideal cipher $\BC$ to set the intermediate variables such that the constraints are fulfilled.
    This is only possible, if the either the output or the input is undetermined by previously fixed variables.
\end{itemize}


\begin{defn}[Collision structure]
Let $\P = (\O, \C)$ be a Linicrypt program with $|\C| = n$.
A \textbf{collision structure} for $\P$ is an index $1 \leq i^* \leq n$, an ordering $(c_1, \dots, c_n)$ of $\C$ for $c_i = (Op_i, \v k_i, \v q_i,\v a_i)$
and a tuple $(dir_{i^*}, \dots, dir_n)$ for $dir_i \in \{\For, \Back\}$,
such that the following two conditions hold:
\begin{enumerate}
\item 
Let $\mathcal{F}_\is = \{
    \v k_1, \dots, \v k_{i^* -1},
    \v q_1, \dots, \v q_{i^* -1},
    \v a_1, \dots, \v a_{i^* -1}
    \}$. One of the following is true:
        \begin{enumerate}
    \item $dir_\is = \For \quad$ and 
        $\quad
            \spn\big(\{ \v k_{i^*}, \v q_{i^*} \}\big) \nsubseteq
            \spn\big( \mathcal{F}_\is \cup \rows\left(\O\right) \big)
        $
    \item $dir_\is = \Back \quad$ and 
        $\quad
            \spn\big(\{ \v k_{i^*}, \v a_{i^*} \}\big) \nsubseteq
            \spn\big( \mathcal{F}_\is \cup \rows\left(\O\right) \big)
        $
\end{enumerate}
\item For all $j \geq i^*$ let $\mathcal{F}_j = \{
    \v k_1, \dots, \v k_{j -1},
    \v q_1, \dots, \v q_{j -1},
    \v a_1, \dots, \v a_{j -1}
    \}$. One of the following is true:
\begin{enumerate}
    \item $dir_j = \For \quad$ and 
        $\quad
        \v a_j \notin \spn\big(
        \mathcal{F}_j
        \cup \{ \v k_j, \v q_j\}
        \cup \rows\left(\O\right)
        \big)
        $
    \item $dir_j = \Back \quad$ and 
        $\quad
        \v q_j \notin \spn\big(
        \mathcal{F}_j
        \cup \{ \v k_j, \v a_j\}
        \cup \rows\left(\O\right)
        \big)
        $
\end{enumerate}
\end{enumerate}
\end{defn}

TODO: Remove this wordy definition.
All the info from here should be integrated into the example

\begin{defn}[Collision structure]
Let $\P = (\O, \C)$ be a Linicrypt program with $|\C| = n$.
A \textbf{collision structure} for $\P$ is an index $1 \leq i^* \leq n$, an ordering $(c_1, \dots, c_n)$ of $\C$ for $c_i = (Op_i, \v k_i, \v q_i,\v a_i)$
and a tuple $(dir_{i^*}, \dots, dir_n)$ for $dir_i \in \{\For, \Back\}$,
such that:
\begin{enumerate}
\item The $i^*$'th constraint is unconstrained by the output of $\P$ and previous fixed constraints.
Let $\mathcal{F} = \{
    \v k_1, \dots, \v k_{i^* -1},
    \v q_1, \dots, \v q_{i^* -1},
    \v a_1, \dots, \v a_{i^* -1}
    \}$
denote the vectors fixed by previous constraints in the ordering.
\begin{enumerate}
    \item if $dir_\is = \For$, the input of the query associated to $c_\is$ is unconstrained:
        \[
            \spn\big(\{ \v k_{i^*}, \v q_{i^*} \}\big) \nsubseteq
            \spn\big( \mathcal{F} \cup \rows\left(\O\right) \big)
        \]
    \item if $dir_\is = \Back$, the output of the query associated to $c_\is$ is unconstrained:
        \[
            \spn\big(\{ \v k_{i^*}, \v a_{i^*} \}\big) \nsubseteq
            \spn\big( \mathcal{F} \cup \rows\left(\O\right) \big)
        \]
\end{enumerate}
\item For all $j \geq i^*$ the constraint $c_j$ does not contradict previous constraints.
Let $\mathcal{F} = \{
    \v k_1, \dots, \v k_{j -1},
    \v q_1, \dots, \v q_{j -1},
    \v a_1, \dots, \v a_{j -1}
    \}$
denote the vectors fixed by previous constraints in the ordering.
\begin{enumerate}
    \item if $dir_j = \For$
        \[
        \v a_j \notin \spn\big( \{
        \v k_1, \dots, \v k_{j -1},
        \v q_1, \dots, \v q_{j -1},
        \v a_1, \dots, \v a_{j -1},
        \} 
        \cup \{ \v k_j, \v q_j\}
        \cup \rows\left(\O\right)
        \big)
        \]
    \item if $dir_j = \Back$
        \[
        \v q_j \notin \spn\big( \{
        \v k_1, \dots, \v k_{j -1},
        \v q_1, \dots, \v q_{j -1},
        \v a_1, \dots, \v a_{j -1},
        \} 
        \cup \{ \v k_j, \v a_j\}
        \cup \rows\left(\O\right)
        \big)
        \]
\end{enumerate}
\end{enumerate}
\end{defn}

\begin{lemma}[Collision structure gives second preimages]
    If collision structure blabla exists for $\P = (\O, \C)$ then blabla with probability 1.
\end{lemma}

\pcb[mode=text]{$\mathsf{FindSecondPreimage}\Big(\Big)$}{
Compute $\vbase$ by executing $\PE(\vx)$  \\
... very similar
}
