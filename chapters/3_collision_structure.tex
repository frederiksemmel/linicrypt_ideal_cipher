\chapter{Revisiting Algebraic Representations}
\label{revisiting_algebraic_representations}

In the previous section we have defined the algebraic representation for a Linicrypt program.
It consists of the input matrix $\Inp \in \F^{k \times \base}$,
the output matrix $\O \in \F^{l \times \base}$
and the set of constraints $\C = \{ (t_i, \Q_i, \va_i) \mid i = 1, \dots, n \}$,
with $\Q_i \in \F^{m_i \times \base}$ and $\va_i \in \F^{1 \times \base}$.
These matrices are such that,
if one constrains $\vbase \in \Fsp$ with $\I\vbase = \m{i_1 & \cdots & i_k}^\top$ for some arbitrary input,
the program can compute the rest of the base variables and output $\O \vbase = \m{o_1 & \cdots & o_l}^\top$.
To simplify the notation we will sometimes write $\vec v k$ for the column vector $\m{v_1 & \cdots & v_k}^\top$.

Our goal for this section is to find a better understanding of algebraic representations and thus also for collision structures.
The algebraic representation suggest
that the input and output matrix play similar roles:
They are linear constraints for the base variables. 
Therefore, collision resistance should be expressible in terms of the ability to solve the constraints $\C$ while setting $\O\vbase$ to some value.  
This idea will be formalized in this section.
A question that arises is:
Which combination of matrices with a structure as described above correspond in essence to a Linicrypt program?
To answer this question,
we need to define what a random oracle constraint is for arbitrary vectors.

\begin{defn}[Random oracle constraint]
A \textbf{random oracle constraint} of dimension $\base$ taking $m$ inputs is a tuple $(t, \v Q, \v a)$ for
$t \in \bin^*$, $\Q \in \F^{m \times \base}$ and $\va \in \F^{1 \times \base}$.
\end{defn}

This definition is a generalization of the definition in the preliminaries.
When constructing the algebraic representation for a Linicrypt program,
we always have the special case that $\va = \m{0&\cdots&0&1&0&\cdots&0}$.
We call $t$ the nonce and refer to $\Q$ as the query matrix and to $\va$ as answer matrix.
If the nonce is the empty string, then we just write $(\Q, \va)$ instead of $(t, \Q, \va)$.
Usually we just say ``a constraint" when the other variables are clear from the context.

The constraint $(t, \Q, \va)$ encodes the relationship via the random oracle between the base variables in a program.
This semantic meaning of the constraints is encoded in the following definition.
\begin{defn}[Solution of constraints]
    Let $\C$ be a set of constraints of dimension $\base$.
    We say a vector $\vv \in \Base$ \textbf{solves} $\C$ if
    $\va \vv = \H(t, \Q \vv)$ for all $(t, \Q, \va) \in \C$.
    Such a $\vv$ is also called a \textbf{solution} of $\C$.
    The set of all solutions to $\C$ is called $\sol(\C)$.
\end{defn}

Because $\H$ is a well-defined function,
and not just any relation,
these requirements extend to the constraints.

\begin{defn}[Well-defined]
A set of (random oracle) constraints $\C$ is \textbf{well-defined} if for any pair of constraints 
$c, c' \in \C$ we have $(t,\Q) = (t', \Q') \implies \va = \va'$.
\end{defn}

When we use a set of constraints, we will implicitly also require that it is well-defined.
Highlighting the function like properties of a constraint $(t, \Q, \va)$,
we will also use the shorthand $(t,\Q) \mapsto \va$, or $\Q \mapsto \va$ if $t$ in case t is an empty string.
This notation was introduced in \cite{EPRINT:HolRosRoy22}.

We want to analyze which sets of constraints have solutions and how these solutions can be computed.
First, every Linicrypt program is a method to solve the constraints $\C$ from its algebraic representation.
If we run a program $\P$ on some input $(\dts i k)$,
it will query the random oracle
and solve each random oracle constraint one by one in the order of the corresponding queries in $\P$.
Let us call the resulting vector containing the values of the base variables in this execution $\vv \in \Fsp$.
Then it is a solution of $\C$ satisfying $\I\vv = (\dts i k)$.

If a set of constraints is well-defined, it might still not  correspond to a valid Linicrypt program.
Consider the set
\[
\C = \left\{ \m{1&0} \mapsto \m{0&1}, \m{0&1} \mapsto \m{1&0}\right\}.
\]
Although it is well-defined,
it would correspond to the two calls to the random oracle that are not compatible with the Linicrypt model.
Specifically, such constraints would correspond to a program which contains the commands
$y = H(x)$ and $x = H(y)$, where $x,y$ are intermediate variables.
There is no order in which these two calls can be executed without doing a reassignment,
which does not exist in Linicrypt.
Still, this set of constraints might have solutions.
For every $x \in \F$ with the property that $x = \H(\H(x))$ the vector $(x, \H(x))$ is a solution of $\C$.
In other words,
the set $\C$ expresses a condition on the base variables which cannot be represented by a Linicrypt program.

In order to characterize which constraints can be solved by a Linicrypt program,
we need to consider the concept of basis change as it was introduced in \cite{C:CarRos16}.

\begin{defn}[Basis change]
    Let $\v B$ be any matrix in $\F^{\base \times \base'}$
    and $\P = (\Inp, \O, \C)$ be a Linicrypt program.
    We define 
    \[
        \C \B \coloneqq \big\{ (t, \Q \B, \va \B) \mid (t, \Q, \va) \in \C \big\}.
    \]
    If $\B$ is invertible,
    we say $\P\B \coloneqq (\Inp \B, \O \B, \C \B)$ is a \textbf{pseudo algebraic representation} of $\P$.
\end{defn}

The concept of base change combines well with the definition of the solution set for some constraints.
Let $\C$ be any set of constraints and consider the basis change matrix $\B$.
Note the following equivalence, which follows from the associativity of the matrix product:
\begin{align*}
&\parbox{30pt}{\phantom{$\iff$}} \vv \textrm{ solves } \C\B \\
&\parbox{30pt}{$\iff$} \H(t, \Q \B \vv) = \va \B \vv \quad \textrm{for all} \quad (t, \Q, \va) \in \C \\
&\parbox{30pt}{$\iff$} \B \vv \textrm{ solves } \C
\end{align*}

This implies that $\B|_{\sol(\C\B)}: \sol(\C\B) \to \sol(\C)$ is a well-defined bijection.

We restrict our goal to characterizing which matrices $\I$ and $\O$ and constraints $\C$ are pseudo algebraic representations of some program $\P$. 
This is interesting because it enables us to choose an arbitrary input matrix for a given set $\C$ and check whether there exists a Linicrypt program which computes solutions to $\C$.
The degeneracy and collision structure attack described in \cite{TCC:McQSwoRos19} are both attacks that can be performed as Linicrypt programs.

First, we describe the form $(\I, \O, \C)$ need to have such that they are the algebraic representation of a Linicrypt program.
\begin{lemma}
\label{is_algebraic_repr}
    Let $\C$ be a finite well-defined set constraints of dimension $\base$ with $|\C| = n$,
    let $\I = \m{\Id_k & 0} \in \F^{k\times\base}$ for $k = \base - n$
    and let $\O \in \F^{l \times \base}$ for some $l \in \NN$.
       
    $(\I, \O, \C)$ is the algebraic representation of a deterministic Linicrypt program $\P$
    if there exists an ordering $(c_1, \dots, c_n)$ of $\C$
    such that for all $i=1, \dots, n$:
    \begin{enumerate}
    \item
    $\rows(\Q_i) \subseteq \rowsp(\dts \ve {k+i-1})$
    \item
    $\va_i = \ve^{k+i}$
    \end{enumerate}
\end{lemma}

\begin{sketch}
    This is only a sketched proof.
    In order to formalize it one first has to formalize the Linicrypt model as is done in \cite{C:CarRos16}.

    Assume there is such an ordering as in the Lemma.
    We will construct the program $\P$.
    First $\P$ will store it's $k$ inputs into intermediate variables,
    thus the input matrix is $\I = \m{\Id_k & 0}$.
    Then each of the random oracle constraints is converted into Linicrypt commands of the form
    $v_{k+i} = H(\dts q {k+i-1})$.
    Condition 1.~from the Lemma ensures that $\dts q {k+i-1}$ are linear combinations only of previously defined intermediate variables.  
    Condition 2.~from the Lemma guarantees that the variable $v_{k+i}$ has not been used previously,
    so the command is a valid Linicrypt command.
    
    Finally, because $k=\base - n$, we have instantiated all base variables,
    and we can therefore output according to $\O$.
\end{sketch}

We will now generalize the condition from this Lemma in order to make it independent of the chosen basis of $\Fsp$.
This leads to the characterization of pseudo algebraic representations.

\begin{defn}[Deterministically Solvable]
\label{def_det_solvable}
    Let $\C$ be a finite well-defined set of constraints of dimension $\base$,
    and let $\I \in \F^{k \times \base}$ for some $k \in \NN$.
    $\C$ is \textbf{deterministically solvable fixing} $\I$ (or fixing $\rowsp(\I)$)
    if there exists an ordering $(c_1, \dots, c_n)$ of $\C$
    such that for all $i=1, \dots, n$:
    \begin{enumerate}
    \item
        \label{solvable1}
        $\rows(\Q_i) \subseteq \rowsp(\I) + \rowsp(\va_1, \dots, \va_{i-1})$
    \item
        \label{solvable2}
        $\va_i \notin \rowsp(\I) + \rowsp(\va_1, \dots, \va_{i-1})$
    \end{enumerate}
    Additionally we require that $\rows(\I) \cup \{\va_1, \dots, \va_n\}$ form a basis of $\F^{1\times\base}$.
    We call $(c_1, \dots, c_n)$ a solution ordering of $\C$ fixing $\I$ (or fixing $\rows(\I)$).
\end{defn}

This is similar to the definition of collision structure in \cite{TCC:McQSwoRos19}.
Indeed, we will use it to combine the collision structure and degeneracy attack from that work.

Intuitively, $\C$ being deterministically solvable fixing $\I$ means the following.
We can constrain $\vv \in \Fsp$ by $\Inp \vv = (i_1, \dots, i_k)$ for an arbitrary input $i_1, \dots, i_k \in \F$,
and then compute each oracle query deterministically (condition \ref{solvable1} in Definition \ref{def_det_solvable}) one by one,
without creating a contradiction (condition \ref{solvable2} in Definition \ref{def_det_solvable}).

If we construct the algebraic representation $(\I, \O, \C)$ of a deterministic Linicrypt program $\P$,
then $\C$ is a deterministically solvable set of constraints fixing $\I$.
Indeed, the solution ordering of $\C$ fixing $\I$
can be exactly the order of the corresponding queries in the execution of $\P$.

\begin{lemma}
\label{alg_rep_det_solvable}
    Let $(\I, \O, \C)$ be the algebraic representation of a deterministic Linicrypt program $\P$ taking $k$ inputs.
    Then $\C$ is deterministically solvable fixing $\I$.
\end{lemma}

\begin{proof}
    Consider a constraint $(t, \Q, \va)$ in $\C$.
    By definition $\va = \ve^i$ for some $i$.
    And $\Q$ has only zeros in the columns including and to the right of column $i$.
    We can sort $\C = \{\dts c n\}$, such that the $\va$'s are sorted.
    Indeed, this is the same order as the associated oracle calls in the execution of $\P$.
    
    Clearly condition \ref{solvable2} from Definition $\ref{def_det_solvable}$ is then fulfilled.
    Because $\P$ is deterministic, 
    the queries to the oracle are linear combinations of input variables and results of previous queries.
    Therefore, condition \ref{solvable1} must be fulfilled.
    Finally, as $\P$ contains no sampling operation, we know $\base = k + n$.
    The associated vectors $\rows(I) = \{ \dts \vi k \}$ to the input variables are linearly independent,
    hence it follows that $\{ \dts \vi k, \dts \va n \}$ is a basis of $\F^{1\times\base}$. 
\end{proof}

The following Lemma shows that the reversed direction also works,
meaning that we can recover a deterministic Linicrypt program from deterministically solvable constraints by applying a basis change.
This characterization up to basis change is not a problem, 
if all security characterization making use of the algebraic representation are invariant under basis change.

\begin{lemma}
\label{det_solvable_lemma}
    Let $\C$ be a set of deterministically solvable constraints fixing $\Inp \in \F^{k \times \base}$ for some $k \in \NN$.
    Let $\O \in \F^{\out \times \base}$ be an arbitrary output matrix for some $\out \in \NN$.
    Then there is a basis change $\B \in \F^{\base \times \base}$
    and a deterministic Linicrypt program $\P$,
    such that $(\Inp \B, \O \B, \C \B)$ is its algebraic representation.
\end{lemma}

\begin{proof}
Let $(c_1, \dots, c_n)$ be the solution ordering of $\C$ fixing $\I = \mtdts \vi k$.
We choose the new basis for $\Frowsp$ as $(\dts \vi k, \dts \va n)$, hence the basis change matrix is defined by
% \[
%     \B^{-1} = 
%     \begin{bmatrix}
% \vi_1^\top &
% \cdots &
% \vi_k^\top &
% \va_1^\top &
% \cdots &
% \va_n^\top
%     \end{bmatrix}^\top.
% \]
\[
    \B^{-1} = 
    \begin{bmatrix}
\vi_1 \\
\vdots \\
\vi_k \\
\va_1 \\
\vdots \\
\va_n
    \end{bmatrix}.
\]

In the following we denote by $\v e_i$ the $i$th canonical row vector in $\Frowsp$.
By definition of $\B$ we have for $i=1, \dots, n$
% TODO this is not compact enough, for its importance
\begin{equation*}
\I = \begin{bmatrix}
    \v e_1 \\
    \vdots \\
    \v e_k
\end{bmatrix} \B^{-1}
\qquad \textrm{and} \qquad
\va_i = \v e_{k+i} \B^{-1},
\end{equation*}
which is equivalent to
\begin{equation*}
\I \B = \begin{bmatrix}
    \v e_1 \\
    \vdots \\
    \v e_k
\end{bmatrix}
\qquad \textrm{and} \qquad
\va_i \B = \v e_{k+i}.
\end{equation*}

Additionally, as $\C$ is solvable via the ordering $(c_1, \dots, c_n)$,
we know that 
\[
\rows(\Q_i \B) \subset \rowsp(\vi_1 \B, \dots, \vi_k \B, \va_1 \B, \dots, \va_{i-1} \B) = \rowsp(\dts \ve k+i-1)
\]

This shows that $(\I\B, \O\B, \C\B)$ fulfills the conditions in Lemma \ref{is_algebraic_repr},
therefore we obtain the corresponding program $\P$ from the statement.
\end{proof}

In order to abstract the questions about deterministic Linicrypt programs to the level of the algebraic representation,
we need the following Lemma and Corollary.

Let $\P = (\I, \O, \C)$ be a deterministic Linicrypt program taking $k$ inputs and returning $l$ outputs.
First we note that one can see a deterministic Linicrypt program as a function from its input space to its output space.
We associate to $\P$ the function $f_\P: \F^k \to \F^l$ which maps $(\dts i k)$ onto the output of $\P(\dts i k)$.

\begin{lemma}
\label{solution_space_bijection}
    Let $\C$ be deterministically solvable fixing some $\I$.
    If we view $\I$ as a function $\Fsp \to \F^k$,
    then $\I|_{\sol(\C)}$ is a bijection.
\end{lemma}

\begin{proof}
    If we set $\O = \Id_\base$
    then we can apply Lemma \ref{det_solvable_lemma} to get a deterministic program $\P$ and basis change $\B$ such that,
    $\P = (\I\B, \B, \C\B)$. 
    The associated function $f_{\P}: \F^k \to \F^\base$ is injective,
    as $\P$ is deterministic.
    By definition of the algebraic representation 
    the vector of the values of the base variables in any execution of $\P$ is in $\sol(\C\B)$.
    Note that the output matrix of $\P$ is $\B$ 
    and $\B|_{\sol(\C\B)}: \sol(\C\B) \to \sol(\C)$ is a bijection.
    Therefore, $f_\P$ is an injective map from $\F^k$ to $\sol(\C)$.

    Now we will show that the input matrix $\I: \Fsp \to \F^k$ is injective
    when restricted to $\sol(\C)$.
    This would then complete the proof, because $\F^k$ is a finite set.
    
    Assume we have $\vv, \vv' \in \sol(\C)$ with $\I\vv = \I\vv'$.
    Also assume that $\C$ is deterministically solvable fixing $\I$
    using the ordering $(\dts c n)$ of $\C$.
    We will inductively prove that $\va_i \vv = \va_i \vv'$ for all $i = 1, \dots, n$.

    Assume this is true for all $ j \leq i$.
    Because of the solution ordering of $\C$ we know that
    $\rows(\Q_{i+1}) \subseteq \rowsp(\I) + \rowsp(\dts \va {i})$. 
    This means there is a $\v \lambda$ such that $\Q_{i+1} = \v \lambda \m{\I^\top & \va_1^\top & \cdots & \va_n^\top}^\top =: \lambda \v A_i$.
    Therefore, we have $\Q_{i+1} \vv' = \vl \v A_i \vv' = \vl \v A_i \vv = \Q_{i+1} \vv$.
    Because $\vv$ and $\vv'$ are solutions to $\C$ we know that
    $\va_{i+1}\vv' = \H(\Q_{i+1}\vv') = \H(\Q_{i+1}\vv) = \va_{i+1}\vv$.

    Because $\rows(\I) \cup \{\va_1, \dots, \va_n\}$ is a basis, 
    the matrix 
    $
    \v C = \m{\I \\ \vdts \va n}
    $
    is invertible.

    Using the assumptions and the results from the induction we have shown that $\v C \vv' = \v C \vv$ and
    therefore $\vv' = \vv$.
    This shows that $\I|_{\sol(\C)}: \sol(\C) \to \F^k$ is injective.
    But because $f_\P: \F^k \to \sol(\C)$ is also injective
    and $\F^k$ is finite,
    both maps are bijections.
\end{proof}

\begin{corollary}
\label{det_solvable_computeable}
    Let $\C$ be deterministically solvable fixing some $\I$.
    For each element in $\F^k$ its inverse under $\I|_{\sol(\C)}$ can be computed with $|\C|$ queries to the random oracle.
\end{corollary}

\begin{proof}
    Consider the function $f_\P$ from the proof of Lemma \ref{solution_space_bijection}.
    It can be computed with $|\C|$ queries to the random oracle.
    Because $(\I\B, \B, \C\B)$ is an algebraic representation,
    we have $\I\B = \Id_k \mid 0$. 
    If we run $\P$ on $(\dts i k)$ we compute the values of the remaining base variables.
    Let us call the vector containing the values of the base variables $\vv \in \Fsp$.
    Clearly $\I\B \vv = (\dts i k)$.
    Also, $\P$ outputs $\B\vv$, so $f_\P\big((\dts i k)\big) = \B\vv$. 
    Therefore, 
    \begin{equation}
    \label{input_is_inversion}
    \I\B \vv = (\I \circ f_\P)\big((\dts i k)\big) = (\dts i k).
    \end{equation}
    As $\I|_{\sol(\C)}$ and $f_\P: \F^k \to \sol(\C)$ are bijective,
    \eqref{input_is_inversion} shows that $\I|_{\sol(\C)}$ and $f_\P$ are inverses of each other.
\end{proof}

The benefit of this bijection is that every solution of $\C$ corresponds to a unique input.
This implies, that any question regarding the input and output of a Linicrypt program can be translated to the level of the algebraic representation and solutions of $\C$.
Explicitly, we can formulate the following corollary.

\begin{corollary}
\label{det_solvable_equiv}
    Let $\P = (\I, \O, \C)$ be a deterministic Linicrypt program.
    The following two statements are equivalent:
    \begin{enumerate}
    \item
        Running $\P$ on input $(i_1, \dots, i_k)$ gives output $(o_1, \dots, o_l)$.
    \item 
        There exists a $\vv$ solving $\C$ such that
        $\I\vv = (\dts i k)$ and
        $\O\vv = (\dts o l)$.
    \end{enumerate}
\end{corollary}

\begin{proof}
    The direction 1.~$\implies$ 2.~is clear, by definition of the algebraic representation.
    
    For the other direction, assume we have such a $\vv \in \Base$.
    If we run $\P$ on $\I\vv = (\dts i k)$ we compute the base variables $\vv' \in \Base$.
    Again by definition of the algebraic representation $\vv'$ is a solution to $\C$ with $\I\vv' = (\dts i k) = \I\vv$.
    By Lemma \ref{solution_space_bijection} $\I|_{\sol(\C)}$ is bijective,
    so we have $\vv = \vv'$.
    Running $\P$ on $(\dts i k)$ gives output $\O\vv' = \O\vv = (\dts o l)$.
\end{proof}

\section{Revisiting Collision Structures}

Using this language we can argue about the invertibility and second preimage resistance of a Linicrypt program.
The goal of this section is to describe collision structures in a more abstract form 
which is easier to adapt to the ideal cipher model.
We will describe collision structures as a program being partially invertible with extra degrees of freedom.
As a stepping stone towards this goal, we start by describing a sufficient condition for a Linicrypt program to be invertible.

\begin{lemma}
\label{inversion}
    $\P = (\I, \O, \C)$ be a deterministic Linicrypt program.
    If $\C$ is deterministically solvable fixing $\O$,
    then $f_\P$ is bijective and $f_\P^{-1}$ is the associated function of a deterministic Linicrypt program which we call $\P^{-1}$.
\end{lemma}

\begin{proof}
Assume $\C$ is deterministically solvable fixing $\O$.
Applying the Lemma \ref{det_solvable_lemma} we get a basis change $\B$
and a Linicrypt program $\P^{-1}$ with algebraic representation $(\O \B, \I \B, \C \B)$.
Note, that $\O$ and $\I$ have the same number of rows, because both $\rows(\I) \cup \{\dts \va n\}$ and
$\rows(\O) \cup \{\dts \va n\}$ are bases of $\F^{1\times\base}$.

Because of the bijection $\B: \sol(\C\B) \to \sol(\C)$ we have the following equivalence:
\begin{gather*}
    \vv \textrm{ solves } \C \textrm{ with }
    \I \vv = \vec i k \textrm{ and } \O \vv = \vec o l \\
    \Updownarrow \\
    \B^{-1} \vv \textrm{ solves } \C \B \textrm{ with }
    \O \B \B^{-1} \vv = \vec o l  \textrm{ and }\I \B \B^{-1} \vv = \vec i k  
\end{gather*}

Combining this equivalence with the one from Lemma \ref{det_solvable_equiv} we get:
Running $\P$ on input $\dts i k$ giving output $\dts o l$ is equivalent to running 
$\P^{-1}$ on input $\dts o l$ with output $\dts i k$.
This means that $f_{\P^{-1}} = f_\P^{-1}$, which is what we needed to prove.
\end{proof}

For example, the Lemma applies to the following program $\P$. 
The proof constructs the algebraic representation of the inverse program $\P^{-1}$.
\begin{pchstack}[center,space=2cm]
    \pcbl[valign=c]{$\P(x,y)$}{
        \pcreturn (y + \H(x), x)
    }
    \pcbl[valign=c]{$\P^{-1}(w,z)$}{
        \pcreturn (z, w - \H(z))
    }
\end{pchstack}

To simplify the notation we extend the linear algebra operators $\rowsp$ and $\ker$
to be able to use them naturally with constraints and sets of constraints.
We define for a constraint $c = (t, \Q, \va)$:
\begin{align*}
\rowsp(c) &= \rowsp(\Q) + \rowsp(\va) \\
\ker(c) &= \ker(\Q) \cap \ker(\va)
\end{align*}
For a set of constraints $\C$ we write:
\begin{align*}
\rowsp(\C) &= \sum_{c \, \in \, \C} \rowsp(c) \\
\ker(\C) &= \bigcap_{c \, \in \, \C} \ker(c)
\end{align*}

\begin{lemma}
\label{collision_structure_implies_attack}
    Let $\P = (\I, \O, \C)$ be a deterministic Linicrypt program.
    Assume we can write $\C = \Ccs \cup \Cfix$ and
    $\Ccs$ is deterministically solvable fixing some $\I_{cs}$ such that
    \begin{equation}
    \label{cs_condition}
        \rowsp(\I_\textrm{cs}) \supsetneq \rowsp(\Cfix) + \rowsp(\O).
    \end{equation}
    Then an attacker can find second preimages with $|\Ccs|$ queries.
    We say a program has a collision structure if it fulfills condition \eqref{cs_condition}.
\end{lemma}

This Lemma is similar to Lemma \ref{inversion} in the sense that both describe a condition for being able to find a $\vv$ solving $\C$ while fixing $\O\vv = \vec o l$.
The difference is that here we consider the easier case (from an attacker's perspective):
We already have a solution $\vv'$ with $\O \vv' = \vec o l$ and only require a different one. 

The definition of collision structure in Lemma \ref{collision_structure_implies_attack}
is slightly more general than the one from \cite{TCC:McQSwoRos19}.
It also includes the case of degeneracy, 
namely it corresponds to choosing $\Ccs = \{\}$ and $\I_\textrm{cs} = \Id_\base$.
Degeneracy means precisely that $\F^{1\times\base} \supset \rowsp(\C) + \rowsp(\O)$.
Note, that it is crucial that the space on the left is $\supsetneq$ and not only $\supseteq$,
as this gives the extra degree of freedom to find a different preimage.
It plays the same role as $\vq_{i^*}$ plays in the definition of collision structure from \cite{TCC:McQSwoRos19}.

Here we only give a proof sketch,
a more formal proof will be given in the next section when we adapt Linicrypt to the ideal cipher model.
\begin{sketch}
Assume we are given a $\vv'$ solving $\C$ such that $\I \vv' = \vec i k$ and $\O \vv' = \vec o l$.
The goal is to find a different $\vv$ solving $\C$ with $\O \vv = \vec o l$.
By \eqref{cs_condition} we can constrain $\vv$ to fulfill $\O \vv = \O \vv'$
as well as $\Q \vv = \Q \vv'$ and $\va\vv = \va\vv'$ for every $(t, \Q, \va) \in \Cfix$.
Even doing so, we still have at least one degree of freedom before we have uniquely determined what
$\I_{cs} \vv$ is.
We can therefore choose these further constraints to be arbitrary such that $\vv \neq \vv'$.
Because $\Ccs$ is deterministically solvable fixing $\I_{cs}$,
we can uniquely determine $\vv$ such that it solves $\Ccs$.
But we chose $\I_{cs} \vv$ in such a way, that $\vv$ also solves $\Cfix$,
hence $\vv$ solves $\C = \Ccs \cup \Cfix$.

Finally, because $\P$ is deterministic $\I$ is injective and therefore
$\I\vv$ is a second preimage to $\I\vv'$ for the Linicrypt program $\P$.
\end{sketch}

The main benefit of this perspective on the collision structure attack is
that it directly describes the collision structure attack as a Linicrypt program.
If we apply Lemma \ref{det_solvable_lemma} to $\Ccs$ while fixing $\I_{cs}$,
the resulting ``attack program" is one taking as input the desired output,
some values determined by the execution on the given preimage,
and some extra values which can be set freely.

The following is the running example from \cite{TCC:McQSwoRos19}.
\pcb[valign=c]{$\P(x,y,z)$}{
    w = \H(x) + \H(z) + y \\
    \pcreturn (\H(w) + x, \H(z))
}
% TODO fix alignment
\begin{align*}
\I    &= \begin{bmatrix}
            1 & 0 & 0 & 0 & 0 & 0 \\
            0 & 1 & 0 & 0 & 0 & 0 \\
            0 & 0 & 1 & 0 & 0 & 0 
            \end{bmatrix} \\
\O    &= \begin{bmatrix}
            1 & 0 & 0 & 0 & 0 & 1 \\
            0 & 0 & 0 & 0 & 1 & 0 
            \end{bmatrix} \\
\vq_1 &= \begin{bmatrix}1 & 0 & 0 & 0 & 0 & 0\end{bmatrix} \\
\va_1 &= \begin{bmatrix}0 & 0 & 0 & 1 & 0 & 0\end{bmatrix} \\
\vq_2 &= \begin{bmatrix}0 & 0 & 1 & 0 & 0 & 0\end{bmatrix} \\
\va_2 &= \begin{bmatrix}0 & 0 & 0 & 0 & 1 & 0\end{bmatrix} \\
\vq_3 &= \begin{bmatrix}0 & 1 & 0 & 1 & 1 & 0\end{bmatrix} \\
\va_3 &= \begin{bmatrix}0 & 0 & 0 & 0 & 0 & 1\end{bmatrix} \\
\end{align*}

We set $\Ccs = \{ c_1, c_3 \}$ and $\Cfix = \{c_2\}$.
Then $(c_3, c_1)$ is a solution ordering for $\Ccs$ fixing 
\[
\I_{cs} = \begin{bmatrix}\O \\ \vq_2 \\ \vq_3 \end{bmatrix}.
\]
Condition \eqref{cs_condition} is fulfilled because
\[
\rowsp(\I_{cs}) = \big(\rowsp(\O) + \rowsp(\Cfix)\big) \oplus \rowsp(\vq_3).
\]
The direct sum means that $\vq_3$ is not in $\rowsp(\O) + \rowsp(\Cfix)$.
Then Lemma \ref{det_solvable_lemma} gives us the transformation $\B$
such that $(\I_{cs} \B, \I \B, \Ccs \B)$
is the algebraic representation of some $\P_{cs}$.
Here we have
% TODO fix alignment
\[
\B = 
\begin{bmatrix}
1 & 0 & 0 & 0 &-1 & 0 \\
0 &-1 & 0 & 1 & 0 &-1 \\
0 & 0 & 1 & 0 & 0 & 0 \\
0 & 0 & 0 & 0 & 0 & 1 \\
0 & 1 & 0 & 0 & 0 & 0 \\
0 & 0 & 0 & 0 & 1 & 0
\end{bmatrix},
\]
which leads to the representation:
\begin{align*}
    % TODO fix this constraints mess
    \C_{cs} \B &= \left\{
        \arraycolsep=2pt
        \begin{array}{l l l}
            \m{0 & 0 & 0 & 1 &\phantom{-}0 & 0} & \mapsto & \m{0 & 0 & 0 & 0 & 1 & 0}, \\[2pt]
            \m{1 & 0 & 0 & 0 &-1 & 0} & \mapsto & \m{0 & 0 & 0 & 0 & 0 & 1}
        \end{array}
    \right\} \\
    \I_\textrm{cs} \B &= \m{\Id_4 & 0} \\
    \I \B &= \m{
1 & 0 & 0 & 0 &-1 & 0 \\
0 &-1 & 0 & 1 & 0 &-1 \\
0 & 0 & 1 & 0 & 0 & 0
    }    
\end{align*}

The corresponding Linicrypt program $\P_{cs}$ to this algebraic representation can be written as:
\pcb[valign=c]{$\P_{cs}(o_1,o_2,q_2, q_3)$}{
    a_3 = \H(q_3) \\
    v = \H(o_1 - a_3) \\
    \pcreturn (o_1 - a_3, q_3 -o_2 - v, q_2)
}

This program computes second preimages to the input $(x,y,z)$
if we set its inputs $(o_1, o_2) = \P(x,y,z)$, $q_2 = z$ and $q_3 \neq H(x) + H(z) + y$ arbitrarily.
