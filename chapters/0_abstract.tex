\begin{abstract}
Linicrypt is a mathematical framework introduced by Carmer and Rosulek (Crypto 2016).
It is used to prove cryptographic properties of programs that only make calls to a random oracle and perform linear operations in a field $\F$.
We introduce new abstractions which allow us to extend Linicrypt to work with ideal ciphers instead of random oracles.
In Linicrypt, the execution of a program is viewed as a vector in a finite-dimensional vector space over $\F$.
We describe the set of such vectors in an alternative way and use this formalism to characterize a weakness regarding collision resistance called a collision structure.
Because of the new level of abstraction,
this characterization is applicable simultaneously in the ideal cipher model and the random oracle model.
We show that one can transform a program with a collision structure into its own attack
by performing a basis change on an algebraic representation of itself and then reversing its input and output.
The characterization is sound and complete in the case of Linicrypt programs making only a single query.
We apply these concepts in deriving an attack taxonomy for the \MD construction obtained from the 64 compression functions
introduced by Preneel, Govaerts, and Vandewalle (Crypto 1993).
\end{abstract}
