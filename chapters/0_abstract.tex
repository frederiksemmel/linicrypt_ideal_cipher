\begin{abstract}
Linicrypt is a proof model used to analyze programs that only make calls to a random oracle and perform linear operations in a field $\F$.
We introduce new abstractions which allow us to extend Linicrypt to work with the ideal ciphers instead of random oracles.
Specifically, given Lincrypt oracle constraints $\C$ of dimension $\base$, we define the set of solutions to $\C$ as a subspace of $\Fsp$.
A Linicrypt program can then be interpreted as a method for finding a solution to $\C$ while fulfilling a given linear constraint.
This allows us to characterize a weakness with regard to collision resistance simultaneously for Linicrypt programs in the ideal cipher model and in the random oracle model.
If a program has this weakness, which we call a collision structure,
one can transform the program into its attack by performing a basis change on the algebraic representation and reversing the roles of the input and output.
The characterization is sound and complete in the case of Linicrypt programs making only a single query.
We apply these concepts in the derivation of an attack taxonomy regarding the \MD construction
for the 64 compression functions introduced by Preneel, Govaerts, and Vandewalle (Crypto 1993).
\end{abstract}
