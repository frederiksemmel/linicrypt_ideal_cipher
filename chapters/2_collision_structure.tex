\chapter{Extending Linicrypt to Ideal Ciphers}

Let $\P$ be a Linicrypt program.
For each query to $\E$ of the form $y = \E(k,x)$ we define the associated constraint
$(\E, \v k, \v x, \v y)$, where $\v k \in \F^\base$ is the row vector corresponding
to $k \in \F$ and similarly for $\v x$ and $\v y$.
Each query to $\D$ of the form $x = \D(k, y)$, is associated with the
constraint $(\D, \v k, \v y, \v x)$

To capture the fact that $\E(k, x) = y$ should be associated to the same
constraint as $\D(k,y) = x$ for the same $k$, $x$ and $y$,
we introduce an equivalence relation on the constraints. 
For all $\v k, \v x, \v y \in \Base$ we define
\[
    (\E, \v k,\v x, \v y) \sim (\D, \v k, \v y, \v x).
\]

The set of constraints $\C$ corresponding to $\P$ is then a subset of
\[
    \left( \{\E, \D\} \times \Base \times \Base \times \Base \right) \Big/\sim
\]


Todo: Include the idea that no constraint with the "same" input queries 
are used twice.

Todo: Maybe scrap the idea of the equivalence relation, it seems to hinder more
than it helps.

Todo: Instead of doing weird things with equivalence relation in Collision structure
definition, explicitly add data of reverse or forward direction.


\begin{defn}[Collision structure]
    Let $\P = (\M, \C)$ be a Linicrypt program.
    A \textbf{collision structure} is an index $i^*$ and a tuple
    $(\c_1, \dots, \c_n)$ for $c_i = (O_i, \v k_i, \v q_i, \v a_i)$ and
    $O_i \in \{\E, \D\}$, such that:
    \begin{enumerate}
        \item $[\c_1], \dots, [\c_n]$ is an ordering of $\C$
        \item The input or output corresponding to the query $c_{i^*}$ can be fixed arbitrarily:
        \[
            \span\big(\{ \v k_{i^*}, \v q_{i^*} \}\big) \nsubseteq
            \span\big( 
            \{
            \v k_1, \dots, \v k_{i^* -1},
            \v q_1, \dots, \v q_{i^* -1},
            \v a_1, \dots, \v a_{i^* -1}
            \}
            \cup \rows\left(\M\right)
            \big)
        \]
        \item For all $j \geq i^*$ the constraint $c_j$ does not contradict previous constraints: 
        \[
            \v a_j \notin \span\big( \{
            \v k_1, \dots, \v k_{j -1},
            \v q_1, \dots, \v q_{j -1},
            \v a_1, \dots, \v a_{j -1},
            \} 
            \cup \{ \v k_j, \v q_j\}
            \cup \rows\left(\M\right)
            \big)
        \]
    \end{enumerate}
\end{defn}
